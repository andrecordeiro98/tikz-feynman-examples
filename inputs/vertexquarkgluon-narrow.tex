% Set this figure's name when externalised 
\tikzsetnextfilename{vertexquarkgluon-narrow} 
%
\begin{tikzpicture}
	
	
	% NOTE: MUST *NOT* INCLUDE A {} LABEL FOR THIS ONE
	%		IF A LABEL IS PROVIDED, THERE WILL BE A WHITE CIRCLE OVER THE VERTEX 
	%		USE \node FOR LABELS, \coordinate FOR NO LABEL
	\coordinate (C) at (0,0); 
	
	% Having specified the origin, 
	% the three endpoints are specified in polar coordinates: (angle:radius)
	
	\node (V1) at (180:3.50cm) {};
	\node (V2) at (+20:3.50cm) {};
	\node (V3) at (-20:3.50cm) {};
	
	%%%%%%%%%%%%%%%%%%%%%%%%%%%%%%%%%%%%%%%%%%%%%%%%%%%%%%%%%%%%%%		
	\begin{feynman}
		
		%Diagram
		\diagram*{
			
			(V1) -- [thick, fermion] (C) -- [thick, fermion] (V3),
			
			(V2) -- [thick, gluon] (C)			
			
		};
		
	\end{feynman}
\end{tikzpicture}