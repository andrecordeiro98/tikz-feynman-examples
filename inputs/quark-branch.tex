% Set this figure's name when externalised 
\tikzsetnextfilename{quark-branch} 
%
\begin{tikzpicture}
	
	
	% NOTE: MUST *NOT* INCLUDE A {} LABEL FOR THIS ONE
	%		IF A LABEL IS PROVIDED, THERE WILL BE A WHITE CIRCLE OVER THE VERTEX 
	%		USE \node FOR LABELS, \coordinate FOR NO LABEL
	\coordinate (C) at (0,0); 
	
	% Having specified the origin, 
	% the three endpoints are specified in polar coordinates: (angle:radius)
	
	\coordinate (V1) at (0.00cm,0) {};
	\coordinate (V2) at (2.00cm,0) {};
	\coordinate (V3) at (4.00cm,0) {};
	\coordinate (V4) at (6.00cm,0) {};
	\coordinate (V5) at (8.00cm,0) {};
	
	% Specify the gluon nodes as polar coordinates from offset
	
	\node (G2) at ($(V2) + (30:2.00cm)$) {};
	\node (G3) at ($(V3) + (30:2.00cm)$) {};
	\node (G4) at ($(V4) + (30:2.00cm)$) {};
	
	%%%%%%%%%%%%%%%%%%%%%%%%%%%%%%%%%%%%%%%%%%%%%%%%%%%%%%%%%%%%%%		
	\begin{feynman}
		
		%Diagram
		\diagram*{
			
			(V1)	-- [thick, fermion] (V2)
					-- [thick, fermion] (V3) 
					-- [thick, fermion] (V4) 
					-- [thick, fermion] (V5),
						
			(G2) -- [thick, gluon] (V2),		
			(G3) -- [thick, gluon] (V3),			
			(G4) -- [thick, gluon] (V4)			
			
		};
	
	\end{feynman}
\end{tikzpicture}