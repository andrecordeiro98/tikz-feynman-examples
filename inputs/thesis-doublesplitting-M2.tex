% Set this figure's name when externalised 
\tikzsetnextfilename{thesis-doublesplitting-M2} 
%
\begin{tikzpicture}

	% ------------------------------------------------
	% Ensure the dimensions of graphic are appropriate
	\useasboundingbox (0, 0.5*\imageh) rectangle (\imagew, -0.5*\imageh);
	
	% NOTE: MUST *NOT* INCLUDE A {} LABEL FOR THIS ONE
	%		IF A LABEL IS PROVIDED, THERE WILL BE A WHITE CIRCLE OVER THE VERTEX 
	\coordinate (Blob) at (1.50cm, -0.75cm);
	
	\path (Blob) +(1.00cm, 0.00cm) coordinate (Mother);	        
    
    % Specify the quark endpoints
    \path (Mother) +(2.50cm,0.00cm) coordinate (Emission1);
    \path (Mother) +(5.00cm,0.00cm) coordinate (Emission2);
    \path (Mother) +(7.50cm,0.00cm) coordinate (Final);

	% Specify gluon endpoints
	% NOTE: Here, we use relative polar coordinates

	% Gluon labeled k2
	\path (Emission1) +(10:5.00cm) coordinate (Gluon2); % 30 degrees, 2.50 cm away from (Emission1)

	% The arrow for momentum k2
	\path (Emission1) 	+(10:3.00cm) coordinate (ArrowStart2);
	\path (ArrowStart2) +(10:2.50cm) coordinate (ArrowEnd2);

	% Gluon labeled k1
	\path (Emission2) +(65:2.50cm) coordinate (Gluon1);

	% The arrow for momentum k1
	\path (Emission2) 	+(65:0.70cm) coordinate (ArrowStart1);
	\path (ArrowStart1) +(65:2.00cm) coordinate (ArrowEnd1);

    
    %%%%%%%%%%%%%%%%%%%%%%%%%%%%%%%%%%%%%%%%%%%%%%%
    \begin{feynman}
        		    	
	% Diagram
	\diagram*{ %ALWAYS END LINES WITH A COMMA
	
	(Mother) 	-- [thick, fermion, momentum'=\( \boldsymbol{ p_{i}^{}} \)] (Emission1)
			 	-- [thick, fermion, momentum'=\( \boldsymbol{ q_{2}^{}} \)] (Emission2)
			 	-- [thick, fermion, momentum'=\( \boldsymbol{ p_{f}^{}} \)] (Final),

	% This separation happens to avoid overlaps
	(ArrowStart2)	-- [white, momentum = \( \boldsymbol{k_{2}^{}} \)] (ArrowEnd2),
	(Gluon2)		-- [thick, gluon] (Emission1),


	(ArrowStart1)	-- [white, momentum = \( \boldsymbol{k_{1}^{}} \)] (ArrowEnd1),
	(Gluon1)		-- [thick, gluon] (Emission2),
	
	};

    % The Blob
	\vertex[blob] (m) at (Blob) {\scalebox{2.0}{$\mathcal{M}^{ }_{\it h}$}}; 
	
    \end{feynman}  
  
\end{tikzpicture}