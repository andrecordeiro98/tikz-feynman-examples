% Set this figure's name when externalised 
\tikzsetnextfilename{partonshowers-singleparton-multiplescatterings} 

\begin{tikzpicture}

\newlength{\QGPlength}\setlength{\QGPlength}{10.50cm}
\newlength{\QGPheight}\setlength{\QGPheight}{4.00cm}
%
% %%%%%%%%%%%%%%%%%%%%%%%%%%%%%%%%%%%%%%%%%%%%%%%%%%%%%%%%%%
% NOTE TO SELF:
% 	IN TEXSTUDIO, COMMENT MULTIPLE LINES AT ONCE WITH CTRL+T
% %%%%%%%%%%%%%%%%%%%%%%%%%%%%%%%%%%%%%%%%%%%%%%%%%%%%%%%%%%
%
	
	% Proper size --- Bounding box
	\useasboundingbox (0.00cm, 2.00cm) rectangle (10.50cm, -2.00cm);
	
	% ------------------------------------------------
	% COMMENT THESE LINES TO SEE THE INDIVIDUAL STEPS
	% ------------------------------------------------	
	
	% QGP Effect
	\path[left color=orange!60,right color=orange!60] (1.75cm, 2.00cm) rectangle (10.50cm, -2.00cm);
	
	
	% NOTE: MUST *NOT* INCLUDE A {} LABEL FOR THIS ONE
	%		IF A LABEL IS PROVIDED, THERE WILL BE A WHITE CIRCLE OVER THE VERTEX 
	%		USE \node FOR LABELS, \coordinate FOR NO LABEL
	\coordinate (C) at (0,0); 
	
	% Having specified the origin, 
	% the three endpoints are specified in polar coordinates: (angle:radius)
	
	\coordinate (S0) at (2.00cm,0.00cm) {};	
	\coordinate (S1) at ($(S0)+(+25:1.50cm)$) {};
	\coordinate (S2) at ($(S1)+(-30:2.50cm)$) {};
	\coordinate (S3) at ($(S2)+(+30:2.50cm)$) {};
	\coordinate (S4) at ($(S3)+(-20:2.50cm)$) {};

	\coordinate (G0) at ($(S0)+(-90:1.00cm)$) {};
	\coordinate (G1) at ($(S1)+(+90:1.00cm)$) {};
	\coordinate (G2) at ($(S2)+(-90:1.00cm)$) {};
	\coordinate (G3) at ($(S3)+(+90:1.00cm)$) {};

	
	%%%%%%%%%%%%%%%%%%%%%%%%%%%%%%%%%%%%%%%%%%%%%%%%%%%%%%%%%%%%%%		
	\begin{feynman}
		
		%Diagram
		\diagram*{
			
			(C) -- [thick, fermion] (S0)
				-- [thick, fermion] (S1)
				-- [thick, fermion] (S2)
				-- [thick, fermion] (S3)
				-- [thick, fermion] (S4),

			(G0) -- [thick, gluon] (S0),				
			(G1) -- [thick, gluon] (S1),
			(G2) -- [thick, gluon] (S2),
			(G3) -- [thick, gluon] (S3),

			
		};
		
	\end{feynman}
\end{tikzpicture}