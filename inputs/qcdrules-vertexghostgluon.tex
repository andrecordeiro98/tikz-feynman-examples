% Set this figure's name when externalised 
\tikzsetnextfilename{qcdrules-vertexghostgluon}
%
\begin{tikzpicture}

	
	% NOTE: MUST *NOT* INCLUDE A {} LABEL FOR THIS ONE
	%		IF A LABEL IS PROVIDED, THERE WILL BE A WHITE CIRCLE OVER THE VERTEX 
	\coordinate (C) at (0,0); 
	
	% Having specified the origin, 
	% the three endpoints are specified in polar coordinates:
	
	\node (V1) at (180:2.50cm) {\(\mu, a\)};
	\node (V2) at (+60:2.50cm) {\(b\)};
	\node (V3) at (-60:2.50cm) {\(c\)};	
	
	%%%%%%%%%%%%%%%%%%%%%%%%%%%%%%%%%%%%%%%%%%%%%%%%%%%%%%%%%	
    \begin{feynman}

    %Diagram
    \diagram*{

    (V1) -- [thick, gluon, momentum' =\({ \boldsymbol{k_{}^{}} }\)
			] (C),

    (C) -- [very thick, ghost, momentum =\({ \boldsymbol{p_{1}^{}} }\)
           ] (V2),

	%momentum' == arrow on the outside
    (V3) -- [very thick, ghost, momentum' =\({ \boldsymbol{p_{2}^{}} }\)
           	] (C),

    };
    
    \end{feynman}

\end{tikzpicture}