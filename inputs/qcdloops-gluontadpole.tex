% Set this figure's name when externalised 
\tikzsetnextfilename{qcdloops-gluontadpole} 
%
\begin{tikzpicture}
	
	\begin{feynman}
	
	% ------------------------------------------------
	% Ensure the dimensions of graphic are appropriate
	%\useasboundingbox (0, 0.7*\loopsh) rectangle (\loopsw, -0.7*\loopsh);
	
	% ------------------------------------------------
	% Relevant coordinates
	\coordinate (Beg) at (0.20cm,0);
	
	\path (Beg) +(2.50cm,0) coordinate (V1);
	\path (V1)  +(2.50cm,0) coordinate (End);
	
	% Loop should start slightly above line
	\path (V1)	+(-0.05cm,0.15cm) coordinate (V1LoopStart);
	
	% Loop Label
	\path (V1LoopStart)	+(-1.15cm,2.50cm) coordinate (LabelStart);
	\path (LabelStart)	+( 2.50cm,0.00cm) coordinate (LabelEnd);
		
	%%%%%%%%%%%%%%%%%%%%%%%%%%%%%%%%%%%%%%%%%		
	%Diagram
	\diagram*{
						
		(V1)	-- [thick, gluon, rmomentum = \(\boldsymbol{k}\)] (Beg),
		(End)	-- [thick, gluon, rmomentum = \(\boldsymbol{k}\)] (V1),
	
		(LabelStart) -- [opacity=0, momentum = \(\boldsymbol{q}\)] (LabelEnd),
	
	};
	
	% Draw the loop
	%\draw[<options>] (x,y) arc (start:stop:radius);
	\draw[thick, gluon] (V1LoopStart) arc (-95:272:1.25cm);
	
	% Draw the vertex dot
	\node[dot, above right = 0.00cm and -0.04cm of V1] (Dot);
		
	\end{feynman}
	
\end{tikzpicture}