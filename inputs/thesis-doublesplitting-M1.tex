% Set this figure's name when externalised 
\tikzsetnextfilename{thesis-doublesplitting-M1} 
%
\begin{tikzpicture}
    
    % ------------------------------------------------
    % Ensure the dimensions of graphic are appropriate
    \useasboundingbox (0, 0.5*\imageh) rectangle (\imagew, -0.5*\imageh);
    
    % NOTE: MUST *NOT* INCLUDE A {} LABEL FOR THIS ONE
    %		IF A LABEL IS PROVIDED, THERE WILL BE A WHITE CIRCLE OVER THE VERTEX 
    \coordinate (Blob) at (1.50cm, -0.75cm);

    \path (Blob) +(1.00cm, 0.00cm) coordinate (Mother);	        
    
    % Specify the quark endpoints
    \path (Mother) +(2.50cm,0.00cm) coordinate (Emission1);
    \path (Mother) +(5.00cm,0.00cm) coordinate (Emission2);
    \path (Mother) +(7.50cm,0.00cm) coordinate (Final);

	% Specify gluon endpoints
	% NOTE: Here, we use relative polar coordinates
	\path (Emission1) +(40:3.00cm) coordinate (Gluon1); % 30 degrees, 2.50 cm away from Emission1
	\path (Emission2) +(40:3.00cm) coordinate (Gluon2); 
    
    %%%%%%%%%%%%%%%%%%%%%%%%%%%%%%%%%%%%%%%%%%%%%%%
    \begin{feynman}

	% Diagram
	\diagram*{ %ALWAYS END LINES WITH A COMMA
	
	(Mother) 	-- [thick, fermion, momentum'=\( \boldsymbol{ p_{i}^{}} \)] (Emission1)
			 	-- [thick, fermion, momentum'=\( \boldsymbol{ q_{1}^{}} \)] (Emission2)
			 	-- [thick, fermion, momentum'=\( \boldsymbol{ p_{f}^{}} \)] (Final),
	
	(Gluon1) 	-- [thick, gluon, rmomentum' = \( \boldsymbol{k_{1}^{}} \)] (Emission1),

	(Gluon2) 	-- [thick, gluon, rmomentum' = \( \boldsymbol{k_{2}^{}} \)] (Emission2),
	
	};
	
    % The Blob
    % Called last so it overlaps with the fermion line
	\vertex[blob] (m) at (Blob) {\scalebox{2.0}{$\mathcal{M}^{ }_{\it h}$}}; 

    \end{feynman}    

\end{tikzpicture}