% Set this figure's name when externalised 
\tikzsetnextfilename{partonshowers-quarkgluonsplitting} 
%
\begin{tikzpicture}
	
	
	% NOTE: MUST *NOT* INCLUDE A {} LABEL FOR THIS ONE
	%		IF A LABEL IS PROVIDED, THERE WILL BE A WHITE CIRCLE OVER THE VERTEX 
	%		USE \node FOR LABELS, \coordinate FOR NO LABEL
	\coordinate (C) at (0,0); 
	
	% Having specified the origin, 
	% the three endpoints are specified in polar coordinates: (angle:radius)
	
	\node (V1) at (180:2.50cm) {};
	\node (V2) at (+30:2.50cm) {};
	\node (V3) at (-30:2.50cm) {};
	
	%%%%%%%%%%%%%%%%%%%%%%%%%%%%%%%%%%%%%%%%%%%%%%%%%%%%%%%%%%%%%%		
	\begin{feynman}
		
		%Diagram
		\diagram*{
			
			(V1) -- [thick, fermion, momentum = \(p\)] (C)
				 -- [thick, fermion, momentum' = \(q\)] (V3),
			
			(V2) -- [thick, gluon, rmomentum' = \(k\)] (C),
			
		};
		
	\end{feynman}
\end{tikzpicture}