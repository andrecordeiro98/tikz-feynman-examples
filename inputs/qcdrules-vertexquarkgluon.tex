% Set this figure's name when externalised 
\tikzsetnextfilename{qcdrules-vertexquarkgluon} 
%
\begin{tikzpicture}

	    
	% NOTE: MUST *NOT* INCLUDE A {} LABEL FOR THIS ONE
	%		IF A LABEL IS PROVIDED, THERE WILL BE A WHITE CIRCLE OVER THE VERTEX 
	%		USE \node FOR LABELS, \coordinate FOR NO LABEL
	\coordinate (C) at (0,0); 
	
	% Having specified the origin, 
	% the three endpoints are specified in polar coordinates: (angle:radius)
	
	\node (V1) at (180:2.50cm) {\(\mu, a\)};
	\node (V2) at (+60:2.50cm) {\(i\)};
	\node (V3) at (-60:2.50cm) {\(j\)};
	
	%%%%%%%%%%%%%%%%%%%%%%%%%%%%%%%%%%%%%%%%%%%%%%%%%%%%%%%%%%%%%%		
    \begin{feynman}

    %Diagram
    \diagram*{

	(V1) -- [thick, gluon] (C),
    
    (V3) -- [thick, fermion] (C)
         -- [thick, fermion] (V2),

    };
    
    \end{feynman}
\end{tikzpicture}