% Set this figure's name when externalised 
\tikzsetnextfilename{thesis-doublesplitting-M3} 
%
\begin{tikzpicture}
	
	% ------------------------------------------------
	% Ensure the dimensions of graphic are appropriate
	\useasboundingbox (0, 0.5*\imageh) rectangle (\imagew, -0.5*\imageh);
	
	% NOTE: MUST *NOT* INCLUDE A {} LABEL FOR THIS ONE
	%		IF A LABEL IS PROVIDED, THERE WILL BE A WHITE CIRCLE OVER THE VERTEX 
	\coordinate (Blob) at (1.50cm, -0.75cm);
	
	\path (Blob) +(1.00cm, 0.00cm) coordinate (Mother);	        
	
	% Specify the quark endpoints
	\path (Mother)	  +(3.00cm,0.00cm)	coordinate (Emission1);
	\path (Emission1) +(25:2.50cm)		coordinate (Emission2);
	\path (Mother)    +(7.00cm,0.00cm)	coordinate (Final);
	
	% Specify gluon endpoints
	% NOTE: Here, we use relative polar coordinates
	\path (Emission2) +(30:2.50cm) coordinate (Gluon1); % 30 degrees, 2.50 cm away from Emission1
	\path (Emission2) +(00:2.50cm) coordinate (Gluon2); 
	
	%%%%%%%%%%%%%%%%%%%%%%%%%%%%%%%%%%%%%%%%%%%%%%%
	\begin{feynman}
		
		% Diagram
		\diagram*{ %ALWAYS END LINES WITH A COMMA
			
			(Mother) 	-- [thick, fermion, momentum'=\( \boldsymbol{ p_{i}^{}} \)] (Emission1)
						-- [thick, fermion, momentum'=\( \boldsymbol{ p_{f}^{}} \)] (Final),

			(Emission2) -- [thick, gluon, rmomentum' = \( \boldsymbol{q_{3}^{}} \)] (Emission1),

			(Gluon1) -- [thick, gluon, rmomentum' = \( \boldsymbol{k_{1}^{}} \)] (Emission2),
			(Gluon2) -- [thick, gluon, rmomentum = \( \boldsymbol{k_{2}^{}} \)] (Emission2),
						
		};
		
		% The Blob
		\vertex[blob] (m) at (Blob) {\scalebox{2.0}{$\mathcal{M}^{ }_{\it h}$}}; 
		
	\end{feynman}    
	
\end{tikzpicture}