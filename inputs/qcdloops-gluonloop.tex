% Set this figure's name when externalised 
\tikzsetnextfilename{qcdloops-gluonloop} 
%
\begin{tikzpicture}

	\begin{feynman}
	
	% ------------------------------------------------
	% Ensure the dimensions of graphic are appropriate
	%\useasboundingbox (0, 0.7*\loopsh) rectangle (\loopsw, -0.7*\loopsh);
	
	% ------------------------------------------------
	% Relevant coordinates
	\coordinate (Beg) at (0.20cm,0);
	
	\path (Beg) +(2.50cm,0) coordinate (V1);
	\path (V1)  +(2.50cm,0) coordinate (V2);
	\path (V2)  +(2.50cm,0) coordinate (End);
	
	\path (V1)				+(0.00cm,1.20cm) coordinate (LabelStartUpper);
	\path (LabelStartUpper)	+(2.50cm,0.00cm) coordinate (LabelEndUpper);

	\path (V1)				+(0.00cm,-1.20cm) coordinate (LabelStartLower);
	\path (LabelStartLower)	+(2.50cm,0.00cm) coordinate (LabelEndLower);

	% ----------------
	% Dotted vertices
	\node[dot] (V1Dot) at (V1) +(0.00cm, 0.00cm);
	\node[dot] (V2Dot) at (V2) +(0.00cm, 0.00cm);

	%%%%%%%%%%%%%%%%%%%%%%%%%%%%%%%%%%%%%%%%%
	%Diagram
	\diagram*{
		
		(Beg)	-- [thick, gluon, momentum' = \(\boldsymbol{k}\)] (V1), 
		(End)	-- [thick, gluon, rmomentum = \(\boldsymbol{k}\)] (V2),
		
		(LabelStartUpper) -- [opacity=0, momentum = \(\boldsymbol{k-\frac{q}{2}}\)] (LabelEndUpper),
		
		(LabelStartLower) -- [opacity=0, rmomentum' = \(\boldsymbol{k+\frac{q}{2}}\)] (LabelEndLower),
		
	};
	
	\draw[thick, gluon] (V1) arc [start angle=180, end angle=0, radius=1.25cm];
	\draw[thick, gluon] (V2) arc [start angle=0, end angle=-180, radius=1.25cm];
	
			
	\end{feynman}
	
\end{tikzpicture}