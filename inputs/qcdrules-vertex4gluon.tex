\tikzsetnextfilename{qcdrules-vertex4gluon} % Set this figure's name when externalised 
%
\begin{tikzpicture}

    
	% NOTE: MUST *NOT* INCLUDE A {} LABEL FOR THIS ONE
	%		IF A LABEL IS PROVIDED, THERE WILL BE A WHITE CIRCLE OVER THE VERTEX 
	%		USE \node FOR LABELS, \coordinate FOR NO LABEL
	\coordinate (C) at (0,0); 
	
	% Having specified the origin, 
	% the three endpoints are specified in polar coordinates:
	
	\node (V1) at (+135:2.56cm) {\(\mu, a\)};
	\node (V2) at ( +45:2.56cm) {\(\nu, b\)};
	\node (V3) at ( -45:2.56cm) {\(\rho,c\)};
	\node (V4) at (-135:2.56cm) {\(\sigma,d\)};
	
	%Pointless, except it extends the image into a square
	%\node (Placeholder) at (0:2.50cm) {\phantom{\(\mu, a\)}};
	
	%%%%%%%%%%%%%%%%%%%%%%%%%%%%%%%%%%%%%%%%%%%%%%%%%%%%%%%%%%%%%%		
    \begin{feynman}

    %Diagram
    \diagram*{

    %(C) -- [thick, gluon, rmomentum =\({ \boldsymbol{p_{1}^{}} }\)] (V1),
    %(C) -- [thick, gluon, rmomentum =\({ \boldsymbol{p_{2}^{}} }\)] (V2),
    %(C) -- [thick, gluon, rmomentum =\({ \boldsymbol{p_{3}^{}} }\)] (V3),
    %(C) -- [thick, gluon, rmomentum =\({ \boldsymbol{p_{4}^{}} }\)] (V4),

    (V1) -- [thick, gluon, momentum' =\({ \boldsymbol{p_{1}^{}} }\)] (C),
	(V2) -- [thick, gluon, momentum' =\({ \boldsymbol{p_{2}^{}} }\)] (C),
	(V3) -- [thick, gluon, momentum' =\({ \boldsymbol{p_{3}^{}} }\)] (C),
	(V4) -- [thick, gluon, momentum' =\({ \boldsymbol{p_{4}^{}} }\)] (C),


    };
    
    \end{feynman}
\end{tikzpicture}