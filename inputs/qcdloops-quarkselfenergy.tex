% Set this figure's name when externalised 
\tikzsetnextfilename{qcdloops-quarkselfenergy} 
%
\begin{tikzpicture}
	
	\begin{feynman}

	% ------------------------------------------------
	% Ensure the dimensions of graphic are appropriate
	%\useasboundingbox (0, 0.7*\loopsh) rectangle (\loopsw, -0.3*\loopsh);
		
	% ------------------------------------------------
	% Relevant coordinates
	\coordinate (Beg) at (0.20cm,0);

	\path (Beg) +(2.50cm,0) coordinate (V1);
	\path (V1)  +(2.50cm,0) coordinate (V2);
	\path (V2)  +(2.50cm,0) coordinate (End);

	\path (V1)			+(0.00cm,1.20cm) coordinate (LabelStart);
	\path (LabelStart)	+(2.50cm,0.00cm) coordinate (LabelEnd);

	% ----------------
	% Dotted vertices
	\node[dot] (V1Dot) at (V1) +(0.00cm, 0.00cm);
	\node[dot] (V2Dot) at (V2) +(0.00cm, 0.00cm);
	
	%%%%%%%%%%%%%%%%%%%%%%%%%%%%%%%%%%%%%%%%%
    %Diagram
	\diagram*{
	
	(Beg)	-- [thick, fermion, momentum' = \(\boldsymbol{p}\)] (V1) 
			-- [thick, fermion, momentum' = \(\boldsymbol{k}\)] (V2)
			-- [thick, fermion, momentum' = \(\boldsymbol{p}\)] (End),
	
	(LabelStart) -- [opacity=0, momentum = \(\boldsymbol{p-k}\)] (LabelEnd)
	
	};

	\draw[thick, gluon] (V1) arc [start angle=180, end angle=0, radius=1.25cm];
		
	%\draw[blue, thick, gluon] (V1) arc [start angle=180, end angle=-180, radius=1.25cm];	
	
	\end{feynman}
	
\end{tikzpicture}